\documentclass[%
%   preprint,
superscriptaddress,
%groupedaddress,
%unsortedaddress,
%runinaddress,
%frontmatterverbose,
%preprint,
%showpacs,preprintnumbers,
%nofootinbib,
%nobibnotes,
%bibnotes,
%twocolumn,
amsmath,amssymb,
aps,
pre,
%prb,
%rmp,
%prstab,
%prstper,
floatfix,
]{revtex4-2}

\usepackage{graphicx}% Include figure files
\usepackage{dcolumn}% Align table columns on decimal point
\usepackage{bm}% bold math
\usepackage{hyperref}% add hypertext capabilities
\usepackage{color}
\usepackage{verbatim}
\graphicspath{{./fig/}}
%\usepackage{epstopdf}
%\usepackage{auto-pst-pdf}


\begin{document}
\title{Toy model for the sawtooth oscillation}
\author{Ralf Mackenbach}
\affiliation{Princeton Plasma Physics Laboratory, Princeton University, Princeton, New Jersey, USA}
\author{C. B. Smiet}
\affiliation{Princeton Plasma Physics Laboratory, Princeton University, Princeton, New Jersey, USA}
\affiliation{Huygens-Kamerlingh Onnes Laboratory, Leiden University, P.O.\ Box 9504, 2300 RA Leiden, The Netherlands}

\begin{abstract}
The Kadomtsev model describes the sawtooth crash as an internal kink driven instability that redistributes the core helical flux. 
It predicts the final state from an initial state, but does not specify the evolution in the crash phase. 
In this paper we present a minimal analytical model with two free parameters that reproduces the predicted changes in topology. 
This model paints an intuitive picture of how the magnetic topology changes during a sawtooth crash. 
Such a model is important for fusion research because the rapidly chaning magnetic field during the crash phase crates electric fields that scatter fast particles. 
\end{abstract}
\maketitle

The sawtooth oscillation occurs when the core current tokamak fusion reactor exceeds a specific threshold. 
Because of the temperature dependent Spitzer resistivity, the plasma current that creates the rotational transform in the reactor preferentially concentrates on the hottest plasma region near the magnetic axis. 
This decreases the safety factor (increases the field line winding) around the core. 
When the parameters of a tokamak discharge are such that the steady-state current density would result in a safety factor below 1 on axis, the sawtooth cycle occurs. 

According to the Kadomtsev model~\cite{kadomtsev1975disruptive}, the core region within the $q=1$ surface becomes susceptible to an internal kink mode~\cite{coppi1976resistive}. 
This pushes out the core region and the $q=1$ surface is broken into an island chain. 
Hot plasma from the core is mixed with cold plasma from outside of the $q=1$ surface into the growing $m/n=1/1$ island. 
This growing island eventually completely displaces the hot core, and the temperature and current are re-distributed. 
Eventually the core increases in temperature again and current diffuses into the core until a new sawtooth crash is triggered. 



\bibliographystyle{naturemag}
\bibliography{refs}

\end{document}
